\section{Introduction}
Robots has been a thing of the future for a very long time being something only depicted in Hollywood movies with not much basis in reality. This fact is slowly changing and now we are seeing robots that are slowly looking like stuff from movies. 
The spot robot from Boston dynamics is the newest example of this. The spot robot is a yellow quadruped robot which literally looks like something taken from a movie.[insert reference to the black mirror episode].

At the same time the construction business needs a robot 

A lot of uses are being found for the robot one of which is in the construction industry. Which leads us to the motivation of this project.


\section{Motivation}
The project is being worked on in collaboration with Dalux which is a software company that makes BIM models (Building information modelling) for entrepreneurs making the construction process more manageable for them. 
Dalux has a desire to help the entrepreneurs get a better overview of what is being done on their construction site on a day to day basis; is a new wall being constructed, are some tubes being assembled etc. 
Dalux essentially wants to give a temporal overview of the construction process to the entrepreneur, while also precisely showing the entrepreneur if the work being done by his employees or contractors are done correctly.
He can use this information to hold his employees or contractors accountable; is the wall being placed where it should be placed, are the tubes being assembled correctly.

At present Dalux has to hire an expensive architect to walk around the construction site and check that everything is going according to plan and that there are not any errors occurring. Needless to say this is a costly affair.
This is where the SPOT robot enters the picture. Because if this process could be automated such that a robot could walk around a construction site instead and do point scans, send the data directly to Dalux’ software, that would be of great benefit for Dalux. 

This might save them a lot of effort and money, since a robot has a big upfront cost but a very small hourly work cost.


\section{Objective}\label{Objective} 
The objective of this project is to make a path-finding algorithm for the spot robot from Boston dynamics on the floor of a given building. The program should be able to get as input CAD data for a one level floor-plan of a building and should from that, output the path that the robot should traverse. The path should take into consideration the dimensions of the robot, such that it does not walk into walls or such that the planned path is not too narrow for the robot.


The path should be in the form of a loop, such that the start and end destinations of the path are the same - such that the robot can be put into the charger the following morning.


The path should furthermore guarantee that the robot visits every room on the floor.
\\\\
\textbf{How will this be done?}
%\subsubsection{How will this be done?}
\\
It makes sense to split this objective into 3 distinct problem definitions:
\begin{enumerate}
    \item \textbf{Making a graph - which we will denote “grid”- graph - from the input data.} From the CAD input we will sample nodes with a given resolution resulting in a discretized floor plan, denoted “grid”-graph. The walls of the floor plan will also be included for visual purposes. The graph is processed in such a way that the robot will always be able to traverse between two nodes in the floor plan if the nodes are connected; either directly or through intermediary nodes.
    
    \item \textbf{Making a subgraph - which we will denote “room”-graph - from the “grid”-graph.}
    The essential nodes that the robot must visit - to get to every room - will be generated, denoted as “room” nodes.

    A subgraph of the “grid”-graph will be generated where only the “room” nodes are included, denoted “room”- graph.
    An optimal graph traversal algorithm (e.g A* ) will be used to find the shortest distance between each node in this subgraph. An important property of this subgraph will be that it is fully connected; this means that all room nodes are connected to each other.

    
    \item \textbf{Making a path that traverses through all “room” nodes in the “grid” graph.} We will find an approximate solution to the traveling salesman problem on the “room” graph. This solution will be used to generate a path on the “grid”-graph where each “room” node is visited at least once.
\end{enumerate}






\begin{comment}

\section{Abstract}
Første problem. En bim model ind og finde fra a til b.
\begin{itemize}
    \item Make a simulation of a particle moving from one point to another
    \item Make the simulation include walls and obstacles
    \item Transfer walls from a real BIM model onto simulation
    \item Make a maximum coverage algorithm to find specific nodes that are most optimal for point clouds
    \item Find an optimal route for the particle to move between maximum coverage nodes.
    \item Start testing on the robot and include 
\end{itemize}



\section*{Objective}
The objective of this project is to program the spot robot from Boston Dynamics
to move around construction sites. The robot should be able to walk around the entire construction site on its own and should – while walking – take pictures (or/and point scans) of the construction site. The purpose of this is that after the construction workers have done their day of work the spot robot should wake up at night and walk around the construction site to see if everything is going according to plan and if all deadlines and milestones are met.

What should be done?
Boundaries: what can I probably not do?
Risk analysis: What parts of the project are going to be difficult





\section*{Previous work}

\section*{Introduction}
Explain what you’re going to do

\section*{Literature review}
(Write about what the field around your research looks like)

\section*{Dataset}
Describe your dataset(s), including literature relevant to that dataset, if any

\section*{Results}
(Tell us what you have found)

\section*{Conclusion}
(Summarize your findings, bring in perspectives)
Notes on what works and didn’t work:

\section*{Litteratur list}

\end{comment}



